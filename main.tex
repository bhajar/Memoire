\documentclass{article}
\usepackage[utf8]{inputenc}
\begin{document}

\begin{center}
{\Large \textbf{L’analyse prédictive et big data: effet de mode, ou réelle amélioration ?}}
\end{center}
\vspace{0.5cm}

\begin{center}
\textbf{Introduction}
\end{center}

\vspace{0.5cm}
\textbf{Mieux comprendre aujourd’hui, c’est prédire demain}
\vspace{0.5cm}

Faire des prédictions et anticiper l’avenir, c’est une envie qui existe sans doute depuis que l’humanité est apparue. Nous cherchons toujours des clés pour mieux lire le futur. Le Big Data, les algorithmes sophistiqués d’analyse de données nous permettent aujourd’hui d’anticiper toujours plus finement de quoi sera fait demain. 
L’analyse prédictive, c’est l’usage statistique des données collectées pour prévoir un comportement sous la forme d’hypothèses valides.Ce type d’analyse ne constitue pas un phénomène entièrement nouveau, mais la forme que prend cette analyse passe des prédictions fondées sur l’expérience, l’intuition et la pensée critique à des prédictions fondées sur l’analyse technologique de données mal ou non exploitées surtout que les statistiques sur la quantité de données créées chaque jour sont étonnantes et les chiffres continuent d’augmenter.
\vspace{0.5cm}

\textbf{Présentation de la problématique :}

\vspace{0.5cm}
Dans le contexte actuel où le consommateur est considéré comme volatile, où la concurrence s’intensifie et où les marchés deviennent saturés, le client devient l’acteur principal de l’entreprise. D’où la naissance un nouvel enjeu stratégique pour les entreprises : la récolte d’informations et leur analyse pour en tirer des renseignements sur leur environnement actuel et sur son évolution. L’information devient alors une source de revenu et de compétitivité voire peut être même un moyen organisationnel. Ces données sont historisées et analysées afin de suivre l’activité passée et présente, et anticiper les évolutions futures et changer les règles du jeu.


L'idée de  ce sujet de ce mémoire m'est venue lors de mon stage de M1 au sein du groupe EDF. Durant ce stage, je me suis rendue compte de certaines problématiques auxquelles mon équipe était confrontée, et l'une des principales était le taux de charge de travail élevé lors des périodes de négociations de contrats.
Pour négocier un contrat, les clients EDF lancent des appels d’offres auxquels les commerciaux répondent.
La réponse à un appel d’offre exige un travail important de plusieurs équipes (plusieurs ateliers, enquêtes et études sont mis en place). De mon côté j’ai eu l’occasion de travailler avec les monteurs d’offres qui se chargent d’aider les commerciaux en établissant le meilleur rapport offre/prix en fonction du besoin du client.
La prédiction de ces périodes de forte demandes de négociations des contrat permettra alors à la direction une bonne gestion de temps et la mise à disposition des ressources humaines nécessaires pour répondre à ces demandes. C’est la raison pour laquelle la direction Marketing fixe comme objectif la mise en place d’un outils de prédiction permettant de donner une estimation du nombre de négociations durant ces périodes.

Curieuse de découvrir dans quelles mesures les analyses prédictives peuvent permettre de rester en tête de la compétition vis à vis des autres concurrents , nous proposons dans cette étude d’analyser les différentes méthodes de datamining et de comparer les outils de prédiction du marché afin de proposer une solution avec ces 3 principaux gains:
\begin{enumerate}
\item Organisationnel: bien dimensionner les équipes, recruter ou déplacer les monteurs d'offres sur d'autres missions , éviter les surcharges...

\item Financier: Sourcing anticipé de l'énergie: Prédiction des volumes de l'énerg-ie qui va nous falloir et donc éviter d'acheter bcp d'énergie qui va rester sans être vendue.

\item Pilotage: Gestion de plan moyen terme de la direction pour les 3 prochaines années, projection des gain pour qu'on puisse piloter et prendre des décisions. aujourd'hui c'est du ressenti, fait avec les moyens de bord.
\end{enumerate}
 
\vspace{0.5cm}
\textbf{Structure du mémoire :}
\vspace{0.5cm}

Afin de répondre au mieux à ce vaste sujet, nous nous focaliserons dans cette étude sur les axes suivants :

\begin{itemize}
\item L’analyse pour la prédiction et ses concepts.

\item Les méthodes théoriques, statistiques et les techniques de datamining pour la prédiction.

\item La comparaison des différents outils de prédiction du marché.
\end{itemize}

  
Après avoir défini les différents concepts, pré-requis pour la bonne compréhe-nsion de cette étude, ces grands principes seront détaillés au travers d'une  approche théorique, puis appliqués à une étude de cas concrète de prédiction de moments de négociation de contrats. Le parallèle entre l'attendu et le constat nous permettra d’apprécier l’intérêt des analyses prédictives dans un tel environnement, tout en apportant des hypothèses d'amélioration sur les problématiques engendrées ou non résolues.

Avec l’aide de nos conclusions comme de nos hypothèses, nous épondrons à la problématique initiale de cette étude tout en offrant une ouverture sur les aspects de la prédiction qui demeurent encore inexplorés.




\end{document}
